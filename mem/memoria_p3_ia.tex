\documentclass[a4paper]{article}

\usepackage[spanish]{babel}
\usepackage[utf8]{inputenc}
\usepackage{amsmath}
\usepackage{graphicx}
\usepackage[colorinlistoftodos]{todonotes}
\usepackage{vmargin}
\usepackage{listings}

\setpapersize{A4}

\setmargins{2.5cm} % margen izquierdo
{1.5cm} % margen superior
{16.5cm} % anchura del texto
{23.42cm} % altura del texto
{10pt} % altura de los encabezados
{1cm} % espacio entre el texto y los encabezados
{0pt} % altura del pie de página
{2cm} % espacio entre el texto y el pie de página

\lstset{breaklines=true, basicstyle=\footnotesize, frame=single}


\title{Práctica 3 \\ Inteligencia Artificial \\ \large Universidad de Zaragoza}

\author{Marcos Ruiz García, 648045}

\date{\today}

\begin{document}
\maketitle

% \tableofcontents

\section{Introducción}
Se nos ha encargado escribir un programa CLIPS que resuelva el problema de las fichas. Para ello, se deberá utilizar el módulo de control para implementar el algoritmo A*. Es importante separar el módulo de control (MAIN) del resto de módulos para una correcta implementación.

\section{Cambios}
A continuación se van a comentar los cambios que se han realizado sobre la plantilla puzzle.clp, la cual ha sido proporcionada para dicho fin.

\subsection{MODULO MAIN}
\begin{enumerate}
\item Se ha añadido el slot coste a la plantilla nodo.
\item Se han modificado tanto el estado-inicial como el estado-final.
\item Se ha hecho de cero la heurística H1.
\begin{lstlisting}
(deffunction MAIN::heuristica ($?estado)
  (bind ?res 0)
    (loop-for-count (?i 1 7)
      (bind ?ficha (nth ?i $?estado))
        (if (eq ?ficha V)
          then
          (loop-for-count (?j 1 ?i)
            (if (eq ?j B)
              then(bind ?res (+ ?res 1))
            )
          )
        )
      )
?res)
\end{lstlisting}
\item Se ha modificado ligeramente la heurística H2 para que sirva para los nuevos estados y se ha comentado ya que la herurística H1 es mejor.
\end{enumerate}
\subsection{MODULO MAIN::INICIAL}
\begin{enumerate}
\item Se ha añadido que el slot coste comience con valor 0.
\item En pasa-el-mejor-a-cerrado ahora tiene en cuenta el coste para que sea un A* y al final se le ha añadido la línea (focus OPERACIONES).
\begin{lstlisting}
(defrule MAIN::pasa-el-mejor-a-cerrado
  ?nodo <- (nodo (clase abierto)(coste ?c1)(heuristica ?h1))
  (not (nodo (clase abierto)(coste ?c2)(heuristica ?h2&:(< (+ ?h2 ?c2) (+ ?h1 ?c1)))))
=>
  (modify ?nodo (clase cerrado))
  (focus OPERACIONES))
\end{lstlisting}
\end{enumerate}
\subsection{MODULO MAIN::OPERACIONES}
\begin{enumerate}
\item Tanto las reglas izquierda como derecha se han modificado para que sean capaces de mover el hueco (H) hasta tres posiciones a la izquierda o hasta tres posiciones a la derecha respectivamente.
\end{enumerate}
\subsection{MODULO RESTRICCIONES}
\begin{enumerate}
\item La regla repeticiones-de-nodo ha sido modificada para que elimine todos los nodos los cuales esten en el mismo estado pero su camino sea mayor.
\end{enumerate}
\subsection{MODULO MAIN::SOLUCION}
\begin{enumerate}
\item La regla reconoce-solucion ha sido modificada para que reconozca correctamente el nuevo estado final.
\item La regla escribe-solucion se ha modificado para que muestre los estados por los que pasa el tablero y el coste de la solución.
\end{enumerate}
\section{Salida}
\begin{lstlisting}
Solucion:
B B B V V H V
B B H V V B V
B B V V H B V
B H V V B B V
H B V V B B V
V B V H B B V
V B V V B B H
V B V V H B B
V H V V B B B
V V V H B B B

Coste: 10
\end{lstlisting}

\section{Conclusión}
Los lenguajes basados en reglas como CLIPS permiten al desarrollador resolver problemas a un más alto nivel que con los lenguajes convencionales (C, Java, C++, ...) permitiendonos centrarnos más en el problema real. Los lenguajes como CLIPS nos permiten representar el conocimiento mediante reglas heurísticas que especifican las acciones a ser ejecutadas dada una determinada situación.

\end{document}


