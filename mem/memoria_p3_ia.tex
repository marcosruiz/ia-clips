\documentclass[a4paper]{article}

\usepackage[spanish]{babel}
\usepackage[utf8]{inputenc}
\usepackage{amsmath}
\usepackage{graphicx}
\usepackage[colorinlistoftodos]{todonotes}
\usepackage{vmargin}
\usepackage{listings}

\setpapersize{A4}

\setmargins{2.5cm} % margen izquierdo
{1.5cm} % margen superior
{16.5cm} % anchura del texto
{23.42cm} % altura del texto
{10pt} % altura de los encabezados
{1cm} % espacio entre el texto y los encabezados
{0pt} % altura del pie de página
{2cm} % espacio entre el texto y el pie de página


\title{Práctica 3 \\ Inteligencia Artificial \\ \large Universidad de Zaragoza}

\author{Marcos Ruiz García, 648045}

\date{\today}

\begin{document}
\maketitle

\tableofcontents

\section{Introducción}
Se nos ha encargado escribir un programa CLIPS que resuelva el problema de las fichas. Para ello, se deberá utilizar el módulo de control para implementar el algoritmo A*. Es importante separar el módulo de control (MAIN) del resto de módulos para una correcta implementación.

\section{Cambios}

\section{Salida}
\begin{lstlisting}
Solucion:
B B B V V H V
B B H V V B V
B B V V H B V
B H V V B B V
H B V V B B V
V B V H B B V
V B V V B B H
V B V V H B B
V H V V B B B
V V V H B B B

Coste: 10
\end{lstlisting}

\section{Conclusión}
Los lenguajes basados en reglas como CLIPS permiten al desarrollador centrarse 

\end{document}
